%%%%%%%%%%%%%%%%%%%%%%%%%%%%%%%%%%%%%%%%%%%%%%%%%%%%%%%%%%%%%%%%%%%%%%
%
% レポートテンプレート
%
% updated 22 Oct, 2018
% last updated 02 Apr, 2021
%
% (c) Tohru TAKADA@UEC
% 各自のレポートに合わせて変更して使ってください.上記の行は残して使うこと.
% 2次配布可です.ご利用は計画的に.
%
%%%%%%%%%%%%%%%%%%%%%%%%%%%%%%%%%%%%%%%%%%%%%%%%%%%%%%%%%%%%%%%%%%%%%%
\documentclass[dvipdfmx,a4paper,10pt]{jarticle}
\usepackage{graphicx}
\usepackage{amsmath}
\usepackage{latexsym}
\usepackage{multirow}
\usepackage{url}
\usepackage[separate-uncertainty]{siunitx}
\usepackage{pdfpages}
\setlength{\textwidth}{165mm} %165mm-marginparwidth
\setlength{\marginparwidth}{40mm}
\setlength{\textheight}{225mm}
\setlength{\topmargin}{-5mm}
\setlength{\oddsidemargin}{-3.5mm}

\def\vector#1{\mbox{\boldmath $#1$}}
\newcommand{\AmSLaTeX}{%
 $\mathcal A$\lower.4ex\hbox{$\!\mathcal M\!$}$\mathcal S$-\LaTeX}
\newcommand{\PS}{{\scshape Post\-Script}}
\def\BibTeX{{\rmfamily B\kern-.05em{\scshape i\kern-.025em b}\kern-.08em
 T\kern-.1667em\lower.7ex\hbox{E}\kern-.125em X}}
\newcommand{\pderiv}[2]{{\partial#1\over\partial#2}}
\newcommand{\deriv}[2]{{{\rm d}#1\over{\rm d}#2}}
\newcommand{\dderiv}[2]{{{\rm d}^2#1\over{\rm d}#2^2}}
\newcommand{\DeLta}{{\mit\Delta}}
\renewcommand{\d}{{\rm d}}
\def\wcaption#1{\caption[]{\parbox[t]{100mm}{#1}}}
\def\rm#1{\mathrm{#1}}
\def\tempC{^\circ \rm{C}}

\makeatletter
%\def\section{\@startsection {section}{1}{\z@}{-3.5ex plus -1ex minus % -.2ex}{2.3ex plus .2ex}{\Large\bf}}
\def\section{\@startsection {section}{1}{\z@}{-3.5ex plus -1ex minus
-.2ex}{2.3ex plus .2ex}{\normalsize\bf}}
\makeatother

\makeatletter
\def\subsection{\@startsection {subsection}{1}{\z@}{-3.5ex plus -1ex minus
-.2ex}{2.3ex plus .2ex}{\normalsize\bf}}
\makeatother

\makeatletter
\def\@seccntformat#1{\@ifundefined{#1@cntformat}%
   {\csname the#1\endcsname\quad}%      default
   {\csname #1@cntformat\endcsname}%    enable individual control
}
\makeatother

% 1行数式はequation
% 複数行はeqnarray
% \deriv は1階微分演算子
% \dderiv は2階微分演算子
% 下に注で書くときは\footnote{}
% 箇条書きはenumerate \setlength{\itemsep}{-2mm} \item ...

\begin{document}
	\begin{center}
		{\Large{\bf レポートタイトル}} \\
		{\bf 電気通信大学 1年\\
		2110000 氏名} \\
		{\bf \number\year 年\number\month 月\number\day 日 作成} \\
		{\bf \number\year 年\number\month 月\number\day 日更新}
	\end{center}
	
	\section{実験の目的}
	
	\section{実験の原理}

	\section{実験方法}
	
	\section{実験結果}

	\section{考察}
	
	\section{感想}

	%参考文献
	\begin{thebibliography}{99}
		\bibitem{bi:1} これこれ
	\end{thebibliography}
	
	
	%%%%%%%%%%%%%%%%%%%%%%%%%%%%%%%%%%%%%%%%%%%%%%%%%%%%%%%%%%%%%%%%%%%%%%
	\appendix
	\setcounter{figure}{0}
	\setcounter{table}{0}
	\numberwithin{equation}{section}
	\renewcommand{\thetable}{\Alph{section}\arabic{table}}
	\renewcommand{\thefigure}{\Alph{section}\arabic{figure}}
	%\def\thesection{付録\Alph{section}}
	\makeatletter 
	\newcommand{\section@cntformat}{付録 \thesection:\ }
	\makeatother
	%%%%%%%%%%%%%%%%%%%%%%%%%%%%%%%%%%%%%%%%%%%%%%%%%%%%%%%%%%%%%%%%%%%%%%
	
	\section{付録}
	
\end{document}

